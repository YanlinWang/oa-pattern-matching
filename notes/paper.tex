%\documentclass[preprint]{sigplanconf}

\documentclass[preprint]{llncs}


\usepackage{color}

\usepackage{amsmath}
\usepackage{stmaryrd}
\usepackage{graphicx}
\usepackage{amssymb}
\usepackage{fancyvrb}
\usepackage{url}
\usepackage{pstricks,pst-node,pst-tree}
\usepackage{theorem}
%% \usepackage{mathpartir}
\usepackage{bbm}
\usepackage{pgf}
\usepackage{multirow}

\usepackage{listings}
\usepackage{verbatim}
\usepackage{graphicx}
\usepackage{wrapfig}

\usepackage[T1]{fontenc}
\usepackage[scaled=0.85]{beramono}

% "define" code highlights for Java and Scala
\lstdefinelanguage{JavaScala}{
  morekeywords={public,int,interface,implements,default,
    abstract,case,catch,class,def,%
    do,else,extends,false,final,finally,%
    for,if,implicit,import,match,mixin,%
    new,null,object,override,package,%
    private,protected,requires,return,sealed,%
    super,this,throw,trait,true,try,%
    type,val,var,while,with,yield},
  otherkeywords={=>,<-,<\%,<:,>:,\#,@},
  sensitive=true,
  morecomment=[l]{//},
  morecomment=[n]{/*}{*/},
  morestring=[b]",
  morestring=[b]',
  morestring=[b]"""
}

\lstset{ %
language=JavaScala,                % choose the language of the code
columns=flexible,
lineskip=-1pt,
basicstyle=\ttfamily\small,       % the size of the fonts that are used for the code
numbers=none,                   % where to put the line-numbers
numberstyle=\ttfamily\tiny,      % the size of the fonts that are used for the line-numbers
stepnumber=1,                   % the step between two line-numbers. If it's 1 each line will be numbered
numbersep=5pt,                  % how far the line-numbers are from the code
backgroundcolor=\color{white},  % choose the background color. You must add \usepackage{color}
showspaces=false,               % show spaces adding particular underscores
showstringspaces=false,         % underline spaces within strings
showtabs=false,                 % show tabs within strings adding particular underscores
morekeywords={var},
%  frame=single,                   % adds a frame around the code
tabsize=2,                  % sets default tabsize to 2 spaces
captionpos=none,                   % sets the caption-position to bottom
breaklines=true,                % sets automatic line breaking
breakatwhitespace=false,        % sets if automatic breaks should only happen at whitespace
title=\lstname,                 % show the filename of files included with \lstinputlisting; also try caption instead of title
escapeinside={(*}{*)},          % if you want to add a comment within your code
keywordstyle=\ttfamily\bfseries,
% commentstyle=\color{Gray},
% stringstyle=\color{Green}
}


%%\usepackage{natbib}
%%\bibpunct();A{},
%%\let\cite=\citep

%include lhs2TeX.fmt
%include lhs2TeX.sty
%include forall.fmt

\pagestyle{plain}

%{\theorembodyfont{\sffamily} \newtheorem{theorem}{Theorem}}
%{\theorembodyfont{\sffamily} \newtheorem{lemma}{Lemma}}
%\newtheorem{theorem}{Theorem}
%\newtheorem{lemma}{Lemma}
%\newenvironment{proof}{\textbf{Proof:\hspace{4mm}}}{$\Box$}
\newcommand{\authornote}[3]{{\color{#2} {\sc #1}: #3}}
\newcommand\william[1]{\authornote{william}{green}{#1}}
%%\newcommand\tom[1]{\authornote{tom}{red}{#1}}
\newcommand\bruno[1]{\authornote{bruno}{red}{#1}}
\newcommand\yanlin[1]{\authornote{yanlin}{green}{#1}}
%%\newcommand\bruno[1]{}

\newcommand{\hl}[1]{\textcolor{red}{#1}}

\newcommand\sem[1]{\llbracket #1 \rrbracket_r}
\newcommand\sems[1]{\llbracket #1 \rrbracket_s}
\newcommand\tsem[1]{\llbracket #1 \rrbracket}
\newcommand{\rbm}[1]{\raisebox{-2.0ex}[0.5ex]{#1}}
\newcommand\nat[0]{\mathbb{N}}
\newcommand\unit[0]{\mathbbm{1}}

%\renewcommand{\mathindent}{0pt}


\begin{comment}
\author{Bruno C. d. S. Oliveira and Yanlin Wang}
\institute{The University of Hong Kong\\
\email{\{bruno,ylwang\}@cs.hku.hk}\\
\authorrunning{Bruno Oliveira and Yanlin Wang}} % abbreviated author list (for running head)
\end{comment}

%%%%%%%%%%%%%%%%%%%%%%%%%%%%%%%%%%%%%%%%%%%%%%%%%%%%%%%%%%%%%%%%%%%%%%%%%%%%%%%%
\begin{document}

\title{oapatternmatching}
%%\subtitle{Solving The Expression Problem with Covariant Return Types}
%% Poor Man's Family Polymorphism

\maketitle

\begin{abstract}
Abstract
\end{abstract}

\section{Introduction}\label{sec:intro}
Introduction

\section{Problem}\label{sec:ep}
We have seen a lot of cool usage of Object Algebras\cite{}, and the
extensibility of OAs allows us to build extensible and composible
systems when dealing with recursive data types. But sometimes for some
definitions we need to do pattern matching (examples will be given
later). Using pattern matching is convenient, if not provided, there
are still ways to do it\cite{}, but then it would be very complicated
to define some terms.

For example, in the MDSLCS paper\cite{Hofer:2010:MDL:1868294.1868307},
the Domain-Specific Language of Regions are defined with the
optimization in the union definition: In the union(reg1,reg2)
operator, they want to figure out whether reg1 or reg2 is the
universal region, if so, do the optimization: the result goes directly
to the universal region, otherwise, do not optimize. Apparently, this
optimization relies on pattern matching, and they use case classes in
Scala to support this.

Although case classes have much benefit in that they comes with
constructors and pattern matching power, the drawback of case classes
is obvious: hard to deal with new cases (possibly need to explain
more). 

Another example is equality function, that checks whether two
expressions are syntactically equal or not. This operations needs to
traverse the two expressions and compare their components, which
apparently needs pattern matching. 

The pattern matching power of case classes is what we don't have in
OAs. For example, if we write the Region DSL using OA, the language
interface would be like:

\begin{lstlisting}
  trait RegionAlg[Region] {
    def Univ() : Region
    def Circle(radius : Double) : Region
    def Union(reg1 : Region, reg2 : Region) : Region
  }
\end{lstlisting}

With this region algebra signature, writing compositional
interpretations like \texttt{EvalRegion} (in
MDSLCS\cite{Hofer:2010:MDL:1868294.1868307}) is easy, it can be
expressed as a normal algebra. But for non-compositional
interpretations like \texttt{OptimizeRegion} that needs pattern
matching, it's hard to express using OA.

So the problem we want to solve is to improve OA with pattern matching
power so that we can solve problems that needs both extensibility and
non-compositional definitions using pattern matching.

\section{Solution}
(Solution shortly explained with the arithmatic expressions example.)

\begin{enumerate}
\item Define \texttt{ExpAlg} interface:
  \begin{lstlisting}
  trait ExpAlg[In, Out] {
    def Lit(x : Int) : Out
    def Add(e1 : In, e2 : In) : Out
  }
  \end{lstlisting}

\item For \texttt{ExpAlg}, define the corresponding
  \texttt{PatternExpAlg} that extends \texttt{ExpAlg} with the type
  parameter \texttt{In} the subtype of \texttt{InvExp} (which give us
  the pattern matching power using Scala's support of Options).
  \begin{lstlisting}
  trait InvExp[Exp] {
    val fromLit : Option[Int]
    val fromAdd : Option[(Exp, Exp)]
  }
  trait PatternExpAlg[In <: InvExp[In], Out] extends ExpAlg[In, Out] {
    object Lit { def unapply(e : In) : Option[Int] = e.fromLit }
    object Add { def unapply(e : In) : Option[(In, In)] = e.fromAdd }
  }
  \end{lstlisting}

\item For any features that needs pattern matching, simply extending
  \texttt{PatternExpAlg} will give us the pattern matching interface
  automatically. For example, in the evaluation operation, we could do
  pattern matching like this:
  \begin{lstlisting}
  trait EvalExpAlg[In <: InvExp[In] with Eval] extends PatternExpAlg[In, Eval] {
    def Lit(x : Int) = new Eval { def eval = x }

    def Add(e1 : In, e2 : In) = new Eval {
      def eval = e1 match {
        case Lit(n)    =>
          System.out.println("Been here"); n + e2.eval
        case Add(_, _) =>
          System.out.println("Been there"); e1.eval + e2.eval
        case _         => e1.eval + e2.eval
      }
    }
  }
  \end{lstlisting}

\item To construct a concrete expression that supports evaluation and
  pattern matching, we need to merge the \texttt{EvalExpAlg} with the
  \texttt{InvExpAlg} below that really does the pattern matching job:
  \begin{lstlisting}
  trait InvExpAlg[In] extends ExpAlg[In, InvExp[In]] {
    def Lit(x : Int) = new InvExp[In] {
      val fromLit = Some(x)
      val fromAdd = None
    }
    def Add(e1 : In, e2 : In) = new InvExp[In] {
      val fromLit = None
      val fromAdd = Some(e1, e2)
    }
  }
  \end{lstlisting}

  The merge operation can be easily defined just for merging
  \texttt{EvalExpAlg} and \texttt{InvExpAlg}, for a generic merge, we
  can use the infrastructure created in FOPwOA
  paper\cite{Oliveira:2013:FPO:2524984.2524987}. (need to explain
  more).


\end{enumerate}

\section{Related Work}\label{sec:related}

\bibliographystyle{splncs}
\bibliography{paper}

\appendix

\end{document}

%%% Local Variables:
%%% mode: latex
%%% TeX-master: "."
%%% End:
